% LINGI2255 - Software Development Project
% Final report
\documentclass[11pt, a4paper]{article}   	% use "amsart" instead of "article" for AMSLaTeX format
\usepackage[utf8]{inputenc}
\usepackage[UKenglish]{babel}
\usepackage{graphicx}

\usepackage{amssymb}
\usepackage{xcolor}
\usepackage{hyperref}
\usepackage{url}
\usepackage{csquotes}
\usepackage{enumitem}
\usepackage{lipsum}
\usepackage{listings}
\lstset{language=Python}

\newcommand{\tbf}[1]{\textbf{#1}}
\newcommand{\tit}[1]{\textit{#1}}
\newcommand{\shellcmd}[1]{\\\indent\indent\texttt{\footnotesize\$ #1}\\}
\newcommand{\vshellcmd}[1]{\\\indent\indent\texttt{\footnotesize(venv)\$ #1}\\}


\title{Brief Article}
\author{The Author}
%\date{}							% Activate to display a given date or no date

\begin{document}
%\maketitle

%%%%%% Section
\section{Introduction}

%%%%%% Section
\section{Product}

\subsection{Features}

%%%%%% Section
\section{Changes from previous reports}

%%%%%% Section
\section{Technical discussion}

\subsection{Virtual Environment}

We use \texttt{pyvenv}, a tool that can create virtual environment in \texttt{Python}, to keep the dependencies required by our project in a single place. It creates a folder which contains all the necessary executables to use the packages that our project will need. To create the virtual environment :
\shellcmd{pyvenv venv}
To activate the environment :
\shellcmd{source venv/bin/activate}
Once your virtual environment installed and activated, you can install the dependencies of \texttt{Care4Care}. All the dependencies are listed in the file requirements.txt. You can install all of them with the following command:
\vshellcmd{pip install -r requirements.txt}

\subsection{Create the database}
Once you're in a proper virtual environment with all dependencies installed, you can create the database. To do so, you can go in the root of our \texttt{Django} project and use the built-in command :
\vshellcmd{./manager.py syncdb}
At the end, you will ask to create a ``super user'', you can create a user ``care4care''. If you want to populate the database with some demo datas, you can run :
\vshellcmd{./manager.py loaddata data\_demo.json}

\subsection{Run the project}
One you're in the virtual environment with all dependencies installed and the dabatase created, you can launch the development server :
\vshellcmd{./manager.py runserver}
When the server is launched, you can surf the website in your browser at \url{http://localhost:8000}. 

\subsection{File hierarchy}
The project is divised in three ``Django applications'' wich are differents folders in our project :
\begin{description}[noitemsep]
\item[- main] contains everything about users management
\item[- branch] contains everything about branch and jobs (demands and offers)
\item[- news] contains everything about the news system
\end{description}

Each of these ``Django applications'' contains typic files/folders :
\begin{description}[noitemsep]
\item[- migrations/] is a folder containing migrations files. This is the Django’s way of propagating changes we make in our models (adding a field, deleting a model, etc.) into our database schema.
\item[- templates/] is a folder containing our templates. A template is simply a text file (ofter HTML). It contains variables, which get replaced with values when the template is evaluated, and tags, which control the logic of the template.
\item[- Templatestags/] is a folder in wich we will extend the template engine by defining custom tags, and then make them available in our templates.
\item[- static/] is a folder in wich we will put every static files (images, javascript, photos, etc.) used by the application.
\item[- \_\_init\_\_.py] used to mark the folder on disk as Python package directory.
\item[- admin.py] contains all the logic for the auto-generated django-admin.
\item[- forms.py] contains all the forms used by the ``Django application'' and the validation logic.
\item[- models.py] contains all the models used by the ``Django application''. Models are the source of information about data. Each attribute of a model represents a database field. 
\item[- tests.py] contains unit tests.
\item[- urls.py] contains every URLs available for the ``Django application''.
\item[- views.py] contains the logic of every views (linked to an URL) for the ``Django application''.
\end{description}

There is some more specific folders in the root of our project that need to be describe :
\begin{description}[noitemsep]
\item[- care4care/] contains the settings.py file where you can define all settings relative to our project.
\item[- templates/] contains templates that are linked to a third-party application.
\item[- static/] contains all static files not related to a specific application but more about the whole project.
\item[- media\_root/] contains every files uploaded by users via the website.
\item[- locale/] contains translations (i18n) of the project.
\end{description}

And there is some more specific files in the root of our project that need to be describe :
\begin{description}[noitemsep]

\item[- make\_locale.sh] generates translations .po files into ``locale/''. These files contains every sentences to be translated.
\item[- compile\_locale.sh] compiles translations .po files into usable translations .mo files.
\item[- manage.py] is generated by Django. This is the command-line utility for administrative tasks. 
\item[- start.sh] will run the project in a gunicorn server for production environment.
\end{description}




%%%%%% Section
\section{Conclusion}

\end{document}
