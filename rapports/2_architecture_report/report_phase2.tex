% LINGI2255 - Software Development Project
% Phase 2 - Architecture report
% Framework choice
\documentclass[11pt, a4paper]{article}   	% use "amsart" instead of "article" for AMSLaTeX format
\usepackage[utf8]{inputenc}
\usepackage[UKenglish]{babel}
%\usepackage[parfill]{parskip}    		% Activate to begin paragraphs with an empty line rather than an indent
\usepackage{graphicx}				% Use pdf, png, jpg, or eps§ with pdflatex; use eps in DVI mode
								% TeX will automatically convert eps --> pdf in pdflatex		
\usepackage{amssymb}
\usepackage{xcolor}
\usepackage{pifont}

%SetFonts

%SetFonts
\newcommand{\tbf}[1]{\textbf{#1}}
\newcommand{\tit}[1]{\textit{#1}}

\newcommand{\yes}{\color{green}{\ding{52}}}
\newcommand{\no}{\color{red}{\ding{56}}}


\title{Brief Article}
\author{The Author}
%\date{}							% Activate to display a given date or no date

\begin{document}
%\maketitle
%\section{}
%\subsection{}

\section{Choice of web application framework}

To build a web application, a framework is necessary.
But there are many available, with different features, which don't make the choice easy.
For this project, we have chosen the Django framework, and in the following we're going to explain why we have done this choice.
Here is a table comparison of the different frameworks we considered :

\bigskip
\begin{tabular}{|c|c|c|c|c|c|}
\hline
\tbf{Framework}		& Django		& Ruby On Rails	& Pyramid	&	other	\\
\hline
\tbf{Language}		& Python		& Ruby			& Python	&	\\
\hline
\tbf{Documentation}	& \yes			& \yes			& \yes		&	\\
\hline
\tbf{Architecture}		& MVC (MTV)	& MVC			& MVC		&	\\
\hline
\tbf{ORM}			& \yes			& \yes			& \no		&	\\
\hline
\tbf{Migration} 		& \yes			& \yes			& \no		&	\\
\hline
\tbf{Admin}			& \yes			& \no			& ?			&	\\
\hline
\end{tabular}


\bigskip
We were looking for a web application framework based on a language that was powerful, while quick to learn for the team members that don't know it yet.
From this, three frameworks stand out from the rest (Python \& Ruby): Django, Ruby On Rails and Pyramid.
But Pyramid has a few drawbacks: no ORM (object-relational mapper) and no migration by default, which are important features for us to manage the database.
So we refined our choice to Django \& RoR.
They both have a lot of great features, including: MVC architecture, an object-relational mapper, migrations, and furthermore good documentation and an active community.
Ruby On Rails is mainly based on the principle \tit{convention-over-configuration}, which means that a lot of things are done automatically, letting the framework manage several things (e.g. interactions between the database, the controller, and the view). But it's not always easy to understand what's really going on underneath.
Django, however, tends to follow the Python principle \tit{explicit is better than implicit}, so that it makes it easier to figure out what the code does.
We have a preference for the second approach.

\medskip
At this point, the choice of the framework relates more to the programming language used.
Python and Ruby have quite a similar learning curve.
Knowing that the members of the team have to learn Python for another course, we can assert that this language is known by the entire team.
But that's not the case for Ruby.
Furthermore, two members of the team already have a good knowledge of the Django framework.
So we could rely on them if we face a problem concerning the framework during the development.


\end{document}  